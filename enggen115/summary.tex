
\documentclass[12pt]{article}
\usepackage{../template/NotesTeX}
\begin{document}

\title{Enggen 115}
\author{Alexander Bailey}
\emailAdd{alexkingstonbailey@gmail.com}
\maketitle
\flushbottom

\section{The Engineering Design Process}
\section{Spatial Visualisation}
\section{Technical Drawings}
\section{Solid Modelling and Drawing with CAD}
In this course, we use the 3D modelling software Autodesk Inventor.
This is a very popular and powerful software with many features.
In the course, we will only achieve a basic working level of knowledge. 
Later courses will build on this.

\marginnote{\textit{USE A MOUSE!}}[-2cm]

\subsection{Starting CAD}
It is often good to start your design with a technical drawing.
Having a comprehensive plan, with all measurements makes it much easier.
The process then becomes less of a design process and more of a `conversion' process.  

\subsection{Inventor Tips}
\begin{theorem*}
  \textbf{Part naming:} Give your part file a logical name, and avoid "space" characters. While you technically can change the name later on, it can cause problems.

  \textbf{Re-orienting your model (or sketch):} There are several different ways to move and rotate your model to view it in the desired orientation. These are all available in the panel on the right of the screen (see below), or you can:  

  \textbf{Pan:} to re-position the view without rotating 

  \begin{itemize}
    \item Hold down the mouse scroll wheel and drag the model.
    \item Hold down F2 and (still holding F2) drag the left mouse button.
  \end{itemize}

  \textbf{Zoom:} to magnify/diminish the size of the view

  \begin{itemize}
    \item Scroll the mouse wheel.
    \item Hold down F3 and drag the left mouse button.
  \end{itemize}

  \textbf{Rotate:} to spin/flip the view
  \begin{itemize}
    \item Hold down shift and the press the scroll wheel on your mouse.
    \item Hold down F4 and drag the left mouse button.
    \item Click on the desired orientation on the top right view cube.
  \end{itemize}
\end{theorem*}

\section{Truss Design}
\end{document}
