\documentclass[a4paper, 12pt]{article}

\usepackage{amsmath, amssymb}

\begin{document}

\title{Dynamics Summary}
\author{Alexander Bailey}

\maketitle

\section*{Kinematics of Particles}
\par\noindent\rule{0.4pt}{0.4pt}
\begin{center}
\subsection*{The 7 Holy Steps}
\end{center}
Here are Hazim's 7 holy steps for kinematics of particles:
\begin{enumerate}
    \item Identify provided information and what is required 
    \item Sketch the motion and identify points of interest on the path 
    \item Choose coordinate system
    \item Construct the kinematic conditions table 
    \item For each interval, identify the type of acceleration and choose the appropriate equations to solve
    \item Solve equations for required information
    \item Check that your answers make sense
\end{enumerate}
\noindent
Ensure that when you integrate it is definite integration (with limits) because we will almost 
always know the limits and it's preferred over indefinite integration. Additionally, ALWAYS draw a FBD for a question
(or maybe a sketch of the motion) and note that you might want to draw a kinematics condition table for every
part of a question as the conditions change.

\begin{center}
\subsection*{Rectilinear Motion}
\end{center}
Also called linear motion, rectilinear motion is motion in 1 dimension i.e a straight line.
These kinematic equations form the basis of rectilinear motion:

\begin{align*}
    v = \frac{ds}{dt} \text{, }
    a = \frac{dv}{dt} \text{, }
    s = v\frac{ds}{dt}\\
\end{align*}
If acceleration is constant, we can derive these equations by integration:
\begin{align}
    v - v_0 = a(t-t_0) \\ 
    v^2 - v_0^2 = 2a(s-s_0) \\ 
    s - s_0 = v_0(t-t_0) + \frac{1}{2}a(t-t_0)^2
\end{align}

The general approach for a rectilinear motion question is detailed in the 7 holy steps but for the solution
steps we will use any of the equations above (and maybe others, depending on the question). Acceleration (a) will 
be one of four things: constant, a function of time, a function of velocity or a function of displacement. Each
of these have different methods.

\begin{enumerate}
    \item Const.: use the above equations for constant velocity to solve 
    \item a(t): solve $a = \frac{dv}{dt}$
    \item a(v): solve $a = \frac{dv}{dt}$ or $vdv = ads$ depending on what you are solving for
    \item a(s): solve $vdv = ads$ 
\end{enumerate}

\begin{center}
\subsection*{Plane Curvilinear Motion}
\end{center}

\end{document}


