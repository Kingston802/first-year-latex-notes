\documentclass[a4paper, 12pt]{article}

\usepackage{amsmath, amssymb}

\begin{document}

\title{Dynamics Summary}
\author{Alexander Bailey}

\maketitle

\section*{Kinematics of Particles}
\par\noindent\rule{0.4pt}{0.4pt}
\begin{center}
\subsection*{The 7 Holy Steps}
\end{center}
Here are Hazim's 7 holy steps for kinematics of particles:
\begin{enumerate}
    \item Identify provided information and what is required 
    \item Sketch the motion and identify points of interest on the path 
    \item Choose coordinate system
    \item Construct the kinematic conditions table 
    \item For each interval, identify the type of acceleration and choose the appropriate equations to solve
    \item Solve equations for required information
    \item Check that your answers make sense
\end{enumerate}
\noindent
Ensure that when you integrate it is definite integration (with limits) because we will almost 
always know the limits and it's preferred over indefinite integration. Additionally, ALWAYS draw a FBD for a question
(or maybe a sketch of the motion) and note that you might want to draw a kinematics condition table for every
part of a question as the conditions change.

\begin{center}
\subsection*{Rectilinear Motion}
\end{center}
Also called linear motion, rectilinear motion is motion in 1 dimension i.e a straight line.
These kinematic equations form the basis of rectilinear motion:

\begin{align*}
    v = \frac{ds}{dt} \text{, }
    a = \frac{dv}{dt} \text{, }
    s = v\frac{ds}{dt}\\
\end{align*}
If acceleration is constant, we can derive these equations by integration:
\begin{align}
    v - v_0 = a(t-t_0) \\ 
    v^2 - v_0^2 = 2a(s-s_0) \\ 
    s - s_0 = v_0(t-t_0) + \frac{1}{2}a(t-t_0)^2
\end{align}

\noindent
The general approach for a rectilinear motion question is detailed in the 7 holy steps but for the solution
steps we will use any of the equations above (and maybe others, depending on the question). Acceleration (a) will 
be one of four things: constant, a function of time, a function of velocity or a function of displacement. Each
of these have different methods.

\begin{enumerate}
    \item Const.: use the above equations for constant velocity to solve 
    \item a(t): solve $a = \frac{dv}{dt}$
    \item a(v): solve $a = \frac{dv}{dt}$ or $vdv = ads$ depending on what you are solving for
    \item a(s): solve $vdv = ads$ 
\end{enumerate}

\begin{center}
\subsection*{Plane Curvilinear Motion}
\end{center}
\subsubsection*{Simple Projectile Motion}
Simple Projectile Motion is for objects with no thrust or other forces. It is 'simple'.
The primary assumptions are: $a_x=0$ and $a_y=-9.81$ ($a_y$ might changed based on where you put the positive y etc).
The holy steps are the same as for rectilinear motion with just a few additional considerations. 
We know must handle 2 dimensions. This means we consider for v (for example) both $v_x$ and $v_y$. 
This will be the same for s, for a etc.

\subsubsection*{Circular Motion}
Circular Motion is a special case of curvilinear motion involving motion about a fixed point with constant radius. These rotational motion equations will be important:
\begin{align}
    v = r\omega \\
    a_n = r \omega^2 = \frac{v^2}{r} \\
    a_t = r\alpha 
\end{align}
\textbf{Normal Tangential Co-ordinates}
\\
\noindent
It is almost always more convenient to use Normal Tangential (n-t) co-ordinates in circular motion problems.
n-t co-ordinates place the origin at the particle, this means that we can't talk about displacement / position but
we can more easily keep track of a particles velocity and acceleration when considering $a_t$ and $a_n$.
\noindent
The normal component is responsible for changing the direction of the vector. 
It will always act towards the center and will always be positive (just look at the equation).
The tangential component is responsible for changing the magnitude of the vector. 
It will always act 'tangentially' i.e touching the circle at the particle and in the \textit{direction of motion}
If the velocity of the particle is constant then the tangential acceleration is 0.

\section*{Kinetics of Particles}

\section*{Relative motion of particles}
\begin{align*}
  \underline{v}_A = \underline{v}_B + \underline{v}_{A/B} 
\end{align*}
The way to remember this is that the first letter is what you are looking for and the second is what you are relative to. 
$v_{A/B}$ is read as "The velocity of A relative to B".
\subsection*{THE SIX HOLY STEPS}
\begin{enumerate}
    \item Identify provided vector quantities and what is required 
    \item State the relative motion to use 
    \item Construct the kinematic conditions table 
    \item Draw the velocity or acceleration diagram 
    \item Solve equations (using sine/cosine or by vector methods) 
    \item Check that your answers make sense
\end{enumerate}
\subsection*{Drawing velocity/acceleration diagrams}
General conventions are on page 51 of the coursebook. Start from O (for origin, which is just some point),
$v_{A/B}$ starts at b and ends at a, try and draw to scale. 

\section*{Kinematics of Rigid Bodies}
There are three types of motion for R.B. (Rigid  Bodies). 
\begin{enumerate}
  \item Pure Translation $\implies$ $\alpha$ = $\omega$ = 0 
  \item Fixed-axis Rotation (F.A.R.) 
  \item General Plane Motion (combination of the above) 
\end{enumerate}

For a rigid body we can only have one $\omega$, $\alpha$ but different linear velocities at points. In this course, 
we only look at velocities as relative accelerations of R.B. is hard. Additionally, these equations could be useful:
\begin{align}
  |\underline{v}_{A/B}| = |\omega\underline{r}_{A/B}| \\
  \underline{v}_{A/B} = \omega\times\underline{r}_{A/B}
\end{align}
The direction of $\underline{v}_{A/B}$ will be normal to  $\underline{r}_{A/B}$ in the direction of $\omega$.
The seven holy steps of kinematics of particles still apply.
The words "Rolling without Slip" imply: if point 'A' is on the wheel for any instant, $\underline{v}_{A/Sr}$ is 0.
Generally, $\underline{v}_{Sr}$ will be 0.

\end{document}


