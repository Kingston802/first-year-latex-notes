\documentclass[a4paper, 12pt]{article}
\usepackage{amsmath, amssymb, parskip, hyperref}
\hypersetup{
    colorlinks=true,
    linkcolor=blue,
    filecolor=magenta,      
    urlcolor=cyan,
}
\urlstyle{same}

\begin{document}
\section*{Convention}
\begin{itemize}
  \item Your workings must be neat and easily understood. 
  \item All calculations must include formulas.
  \item All values must be stated in your solutions.
  \item Assumptions must be stated in your solutions.
  \item Pay attention to small details, such as the difference between $\dot{W}$ and $W$
  \item All explanations can be as simple as bullet point style.
  \item Significant figures must be determined and stated next to the result. 
  \item Use the sign conventions of the course. i.e. when work leaves a system, it is negative.
  \item $\Delta$ in this course means `initial - final'.
  \item Draw block diagrams
  \item Define system boundaries
\end{itemize}
\section*{Systems, DoF, Energy}
L1, L2, L3
\begin{itemize}
  \item Definitions and units 
  \item System Principles
  \begin{itemize}
    \item Accumulation
    \item Open/Closed systems
    \item Block Diagrams
    \item Energy Transfer via heat
    \begin{itemize}
      \item Conduction
      \item Convection 
      \item Radiation 
    \end{itemize}
    \item Degrees-of-freedom analysis
    \item Mass Balances
  \end{itemize}
\end{itemize}
Mass Balance:

$M = \text{in} - \text{out}$

{\scriptsize where $M$ is the accumulation of the system\/}

$\sum \dot{m}_{\text{in}} = \sum \dot{m}_{\text{out}}$

{\scriptsize where $\dot{m}$ is the mass flow rate\/}

DoF Analysis:
$n_{\text{dof}} = n_{\text{unknowns}} - n_{\text{ind. eq}} - n_{\text{other eq.}}$

{\scriptsize $n_{\text{dof}}=0$ means the problem is well defined \/}

Total internal energy:
$\Delta U = \Delta E - \Delta E_{\text{kin}} - \Delta E_{\text{pot}}$

{\scriptsize Remember $\Delta$ is initial - final in this course\/}

Enthalpy:
$H = U + pV$

{\scriptsize where $H$ is enthalpy, $U$ is internal energy, $p$ is pressure and $V$ is volume\/}


\section*{Wind Power}
L9, L10, L11
\begin{itemize}
  \item Social license to operate
  \item Not in my backyard. Wind power is very loud/ugly.
  \item Working principle of a turbine
  \item creativity in wind
\end{itemize}

\section*{Creativity}
L6, L10
\begin{itemize}
  \item Bend, Break, Blend
\end{itemize}
\subsection*{Creativity in wind}
\begin{itemize}
  \item Offshore wind
  \item Large turbines
  \item Flexible blades (inspired by insect)
\end{itemize}

\section*{Hydro}
L4, L6
\begin{itemize}
  \item Renewable Energy Sources
  \item Energy transfer through shaft work
  \item Simplified mass and energy balance to find shaft work 
  \item Calculation of power output through the potential kinetic energy per unit time of water 
  \item Micro Hydro
  \item Power density (electrical power generated per horizontal m\textsuperscript{2})
\end{itemize}

\section*{Solar}
L12, L13
\begin{itemize}
  \item Solar as a renewable energy source
  \item Power density of solar is relatively low
  \item Application of energy balance 
\end{itemize}
% $P_{\text{in}} - P_{\text{max}}$

$E_{\text{photon}} = \frac{hC}{\lambda}$

Calculating properties of solar plants:

$\eta_{\text{max}} = \frac{P_\text{max}}{P_\text{inc}}$

$P_\text{max} = I_\text{max} \times V_\text{max}$

$A = \frac{P_\text{required}}{P_\text{max}}$

{\scriptsize where $P$ is power, $V$ is voltage, $I$ is current, $A$ is cell area, $\eta$ is Efficiency \/}

\newpage
\section*{Batteries}
L14, L15, L16
\begin{itemize}
  \item Battery as energy storage solutions
  \item Working principles of
  \begin{itemize}
    \item Galvanic Cell
    \item PbA battery 
    \item Li-ion battery 
  \end{itemize}
  \item Redox reactions, cell notion
  \item OIL RIG
  \item AnOX \& REDcat
  \item Comparing values of Gibbs free energy shows where it is stored
\end{itemize}
Calculation of Gibbs free energy (if $\Delta_r G^\circ > 0$ then spontaneous)

$\Delta_r G^\circ = \sum G^\circ_\text{products} - \sum G^\circ_\text{reactants}$

EMF of a cell: $E^\circ_\text{cell} = E^\circ_\text{red} + E^\circ_\text{ox}$

{\scriptsize $E^\circ$ = EMF = voltage = cell potential \/}

Current equation: $It = v_enF$

{\scriptsize where $I$ is the current during time $t$, $v_e$ is the number of electrons transferred, $n$ is the number of moles and $F=96500$ \/}

\newpage
\section*{Ethics}
L L17
\subsection*{Resource Management Act}
Sections 5,6,7,8
\subsection*{Ethics/Batteries}
What materials go into a battery (or a solar panel) and where do they come from

\href{https://www.engineeringnz.org/resources/code-ethical-conduct/}{Engineering NZ code of ethical conduct}

\newpage
\section*{Geothermal}
L24, L25, L26, L29
\begin{itemize}
  \item Difference between a dry steam, flash and binary plant 
  \item Fluid carries heat
  \item Porosity of rock determines the amount of fluid in the rock
  \item Permeability, Darcy's law for fluid flow 
  \item The use of thermal energy not converted to electricity (direct use)
\end{itemize}
Stored heat equation:  
$  Q = Ah\rho C(T_r-T_o) $

{\scriptsize where $Q$ is the stored heat energy, $A$ is the area of the plant, $h$ is the depth of the plant, $\rho$ the density of the rock, $C$ is the specific heat capacity of the rock and $T_r$ and $T_o$ are the reservoir and lowest temperatures\/}

Heat energy in the fluid:
$  Q_{\text{fluid}} = \phi (h_{\text{fluid}}-h_o) $

{\scriptsize where $Q_{\text{fluid}}$ is the heat energy stored in 1m\textsuperscript{3} of rock, $\phi$ is porosity and $h$ is enthalpy\/}

Darcy's Law:
$ Q = \frac{kA}{\mu} \frac{dp}{dx} $

{\scriptsize where k is permeability, A is area, $\mu$ the dynamic viscosity and $\frac{dp}{dx}$ the pressure gradient\/}

\section*{Heat Engines}
L21
\begin{itemize}
  \item Carnot cycle (ideal world)
  \item Rankine cycle (real world, power plants)
  \item Working principle:
\end{itemize}
4-step cyclic process where working fluid is continuously vaporized and condensed 
to run a steam turbine that extracts shaft work.
\begin{itemize}
  \item Steam tables and phase transition diagram
  \item Application of mass and energy balance
\end{itemize}

\textit{Energy Balance}
\begin{equation*}
  \Delta \dot{E}_{\text{System}} = \dot{E}_{\text{in}} - \dot{E}_{\text{out}}  
\end{equation*}
\begin{equation*}
0 = \sum[\dot{m}_{\text{in}}(\frac{1}{2} v_{\text{in}}^2+gh_{\text{in}} + \hat{H}_{\text{in}})] - \sum[\dot{m}_{\text{out}}(\frac{1}{2} v_{\text{out}}^2+gh_{\text{out}} + \hat{H}_{\text{out}})] + \dot{Q} + \dot{W}_{\text{S}} + \dot{W}_{\text{EC}}
\end{equation*}

\section*{Biofuels}
L22, L23
\begin{itemize}
  \item Biomass from biofuels
  \item Working principle of a CHP plant
  \item Terminology, formation of biomass and chemical reactions
  \item Application of mass and energy balances
  \item Ideal Gas Law, Gibbs Free Energy
  \item Entropy
\end{itemize}

\section*{Decision Making}
L27, L28
\begin{itemize}
  \item Multicriteria Decision Making
  \item Lexographic Method
  \item Dominance Graphs
  \item Probability Distributions
\end{itemize}
\end{document}


