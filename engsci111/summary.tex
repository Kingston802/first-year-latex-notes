\documentclass[12pt] {article}
\usepackage{amsmath}
\begin{document}

\title{Modelling Process Notes}
\author{Alexander Bailey}
\maketitle

\section*{Modelling Process}
\subsection*{Factors}
Anything that will have some effect on the your calculations e.g. drag, friction, mass, buoyancy, area.
Note that separate things qualities will count separately e.g. a triangle's area and a square's.
\subsection*{Assumptions}
A statement that makes the problem simpler - cancels factors. 
e.g. 'Total length of wood is not reduced when it is cut'
or 'There are no significant currents' 
\subsection*{Precise Problem Statement}
"Given [KEY FACTORS AND ASSUMPTIONS], Find [THE VALUE YOU'RE ASKED TO FIND]"
\subsection*{Formulating a Model}
$x \propto y,1/z \implies x = \frac{ky}{z} \\

\subsection*{Modelling Forces}
Use Newton's 2nd law, subtract negative forces and add positive ones. Use the ones from the list.

\section{Ordinary Differential Equations}
Ordinary differential equations are equations containing one or more functions of one independent variable. 
You can recognise an ODE from a PDE (partial differential equation) because a PDE will contain $\partial$ (pronounced 'del')
and ODEs have standard 'd'. $\frac{dy}{dx}$ means $y$ is the dependent variable and $x$ is the independent variable. $\frac{dx}{dt}$ $x$ is dependent, $t$ is independent.

\subsection*{Properties}
\subsubsection*{Order}
Highest derivative (also equal to number of values needed to find a particular solution) e.g.
\begin{align*}
    &\frac{dy}{dx} = 5x \text{ 1st Order} \\
    &\frac{d^4y}{dx^4} = \frac{dy}{dx} + 2 \text{ 4th Order} \\
\end{align*}

\subsubsection*{Linear}
Involves only derivatives of y and terms of y to the 1st power e.g. ONLY $\frac{dy}{dx}$, $y$ etc.
\begin{align*}
    &\frac{d^4y}{dx^4} + \frac{dy}{dx} = 2 \text{ is linear} \\
    &\frac{dy}{dx} = 2y + 3 \text{ is linear} \\
\end{align*}

\subsubsection*{Homogeneity} 
If all (non-zero) terms involve the dependent variable then the equation is homogeneous
\begin{align*}
    &\frac{dx}{dt} = x \text{ is homogeneous} \\
    &\frac{dy}{dx} = 2y + 3 \text{ is not homogeneous (3 does involve x)} \\
\end{align*}

\subsection*{Forming Differential Equations}
In typical exam questions there are few points at which you will form a differential equation: modelling a set 
of forces in the typical modelling questions, using proportionality or previous knowledge. Typically the modelling 
questions will use Newton's 2nd law which states $\sum F = ma$ and then you can sum the forces and use it to find
mass/acceleration (or their derivatives). 

\subsection*{Solving Differential Equations}
\begin{enumerate} 
    \item Direct Integration
    \item Separation of Variables 
    \item Euler's Method    
    \item Integrating Factor
\end{enumerate}

\section{Probability}

\end{document}
