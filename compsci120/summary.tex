\documentclass[12pt] {article}
\usepackage{amsmath, amssymb}
\newcommand{\Mod}[1]{\ \mathrm{mod}\ #1}
\begin{document}

\title{Compsci 120 Notes}
\author{Alexander Bailey}
\maketitle

\section*{Modulo}
Modulo can be considered the 'remainder operator'. It is represented here with '$\Mod$'
but appears (in the text book and elsewhere as '$\%$')
\subsection*{Congruence}
\begin{equation*}
  a \Mod n \equiv b 
  \implies a \Mod n = b \Mod n
\end{equation*}
\begin{align*}
  a \Mod n = b \Mod n \\
  \implies a-b = nk, k \in \mathbb{Z} \\
\end{align*}
This gives us the operations:
\begin{align}
  &(a+c) \Mod n = (b+d) \Mod n \\
  &(ac) \Mod n = (bd) \Mod n \\
  &(a^k) \Mod n = (b^k) \Mod n 
\end{align}
\subsection*{Finding the last digit}
The last digit of any decimal number a can be found by $a \Mod 10$ for example,
\begin{align*}
  &\text{Finding the last digit of } 213047^{129314} \\
  &213047^{129314} \Mod 10 \\
  &\text{Observe, } 213047 \Mod 10 = 7 \\
  &\implies  213047^{129314} \Mod 10 = 7^{129314} \Mod 10 \text{ (from eqn. 3)} \\
  &\text{Notice, } (7^2) \Mod 10 = 49 \Mod 10 = 9 \\
  &\text{Notice, } (7^2 \cdot 7^2) \Mod 10 = 81 \Mod 10 = 1 \\
  &\text{So for any k, } (7^{4k}) \Mod 10 = (7^2 \cdot 7^{4k}) \Mod 10 = 1 \\
  &\text{Because } 129314  = 129300+12+2 \\
  &\text{We have } (7^{129314}) \Mod 10 = 9 \\
  &\text{So the last character is 9}
\end{align*}


\section*{Probability}
Ordered means that order matters e.g. a pattern of letters. With repetitions means that repeated values will have 
separate outcomes. 
\subsection*{Ordered Choice WITH Repetitions}
$n^k$
\subsection*{Unordered Choice WITH Repetitions}
$\frac{(k-1+n)!}{k!(n+k1)!} = {n + k -1 \choose k}$ 
\subsection*{Ordered Choice without Repetitions}
$\frac{n!}{k!}$
\subsection*{Unordered Choice without Repetitions}
$\frac{n!}{k!(n-k)!} = {n \choose k}$

\section*{Functions}
For a function to exist the domain of g must equal the co-domain of f

\section*{Graphs}

\section*{Proofs}
\subsection*{Direct Proof}
A direct proof is when you take simple known facts, axioms and theorems and 
work them algebraically until you prove the thing you are looking for is true or false.
For statements like 'p implies q' or 'if p then q' we can assume p is true and use 
our known facts/algebra to prove q.
\subsection*{Proof by Contradiction}
Say we have a fact p we want to prove is true. We assume p is false then reach
a contradiction so in fact p \underline{must} be true. Say we want to prove 
p implies q. Then we can assume p is true, but q is false. Then we reach a contradiction.
\subsection*{Proof by Cases}
Say we want to prove a statement in the form 'for all numbers/graphs/etc' but 
the 'numbers/graphs/etc' can be thought of as several families so we deal with each 
separately in different 'cases'. Eg. we could consider odd \& even numbers.
\subsection*{Proof by Construction}
Proof by construction is a bit of an odd ball but can be brought down to simply 
producing an example of what you are trying to prove. If the statement is 
of the form "there exists x" it is a valid proof to just produce a numerical/physical
example otherwise you must work in general i.e "a graph with n vertices".
\subsection*{Proof by Induction}
\end{document}
