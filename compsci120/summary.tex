\documentclass[12pt] {article}
\usepackage{amsmath}
\begin{document}

\title{Modelling Process Notes}
\author{Alexander Bailey}
\maketitle

\section*{Modulo}

\section*{Probability}
Ordered means that order matters e.g. a pattern of letters. With repetitions means that repeated values will have 
separate outcomes. 
\subsection*{Ordered Choice WITH Repetitions}
$n^k$
\subsection*{Unordered Choice WITH Repetitions}
$\frac{(k-1+n)!}{k!(n+k1)!} = {n + k -1 \choose k}$ 
\subsection*{Ordered Choice without Repetitions}
$\frac{n!}{k!}$
\subsection*{Unordered Choice without Repetitions}
$\frac{n!}{k!(n-k)!} = {n \choose k}$

\section*{Functions}
For a function to exist the domain of g must equal the co-domain of f

\end{document}
